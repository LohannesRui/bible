% 标题
\chapter*{创世纪引言}

% 右页眉
\rhead{创世纪(创)引言}

“梅瑟五书”每部的名称,\uline{犹太}人皆以每书的首句首字为名。自\uline{希腊}七十贤士译经以来,皆以每书的内容大意命名。“梅瑟五书”的第一部名为《创世纪》,因为本书所记载的是有关天地万物的创造,人类的太古史和\uline{以色列}民族的起源。但本书并不是以科学的论点和近代史学家的方法来记述,而是本着宗教的观点来说明救赎史的开端。在这救赎史中,依照天主的计划,\uline{以色列}民族在万民中占着最重要的角色,因此作者也只着重于这个民族的历史。

本书的前编(1-11章),是救世史的前导,说明天主是整个天地万物的创造者,和全人类历史的领导者;并指出\uline{以色列}人与其他民族的关系。原祖父母虽然背命,惹下了滔天大祸,后继的人们也多半背弃了天主,如洪水和\uline{巴贝耳}塔时代的人,但因为人是按天主的肖像造成的,天主决不愿将全人类完全抛弃,所以在后编内(12-50章),作者便记述天主怎样拣选了一位信仰坚定,服从听命的人——\uline{亚巴郎},怎样向他起誓,立他为一个新民族的始祖,即将来要成为天主选民的民族的始祖,许下因他和他的后裔,天下万民将要获得祝福(22:18),由他的后裔中要生出一位“应得权杖,万民都要归顺他”(49:10),他要使“救恩达于地极”的后裔(\uwave{依}49:6)。在记述\uline{亚巴郎}、\uline{依撒格}、\uline{雅各伯}和\uline{若瑟}的事迹中,作者一再证明天主怎样特殊地照顾了他们,以准备救赎人类的道路。