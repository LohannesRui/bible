% 标题
\chapter*{费肋孟书引言}

% 右页眉
\rhead{费肋孟书(费)引言}

\uline{费肋孟}原是\uline{哥罗森}城的富翁,他大概是在\uline{厄弗所}(\uwave{宗}19:10)直接为\uline{保禄}所归化的(19节)。他信教后,表现了非凡的信德与爱德,竟把自己的家献出,作为信友集会及举行圣祭之所(2,4-7节)。他既是富户人家,按当时的社会制度,也蓄养了许多奴隶,其中有一名叫\uline{敖乃息摩}的,尚未信教,作了一件对不起主人的事(大约偷了财物),因而畏罪逃亡,出走远方。

\uline{敖乃息摩}逃到了\uline{罗马},找到了正被囚的\uline{保禄}。\uline{保禄}为保护他,起初本想留下他服侍自己(13节),但后来因\uline{敖}氏已领洗入教(10,11两节),决意叫他跟\uline{提希苛}(\uline{保禄}致《哥罗森书》即由他带去),回到他的主人那里,因而写了这封保荐他的短信,求\uline{费肋孟}不但不要处罚他,而且还应以“弟兄”之谊接待他(16节)。这封短信,犹如在《厄弗所书引言》中所提过的,应是\uline{保禄}在63年于\uline{罗马}写成的。

这封优美的私人函件,篇幅极短,很可能全是\uline{保禄}亲笔所写。信内措词造句,委婉动人,务求达到目的;同时这封富于热情的短信也将\uline{保禄}内心的爱情,活显于纸上,实是不可多得的杰作。

这封私人函件,因为涉及了奴隶制度的社会问题,所以对教会以及社会发生了极大的影响。当基督教会出现于世时,经济与社会生活全系于奴隶。虽则如此,但当时的人却不以奴隶为人,而只视作货物。\uline{基督}的教义一传于世,开始了一种新的气象;所以这封短信实可称为“基督自由的宣言”(参阅\uwave{迦}3:27、28;\uwave{格}前7:20-22;\uwave{弗}6:5-9;\uwave{哥}3:11,4:1)。它虽不曾把奴隶制度立即废除,但由于基督教义的逐渐推进,把奴隶制度终归消灭。