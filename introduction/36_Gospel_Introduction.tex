% 标题
\chapter*{福音总论}

% 右页眉
\rhead{福音总论}

“福音”一词,按其字意,原指“喜讯”;但按《新约》作者采用此词的意义来说,乃是指天主子\uline{耶稣}降生为人,从天上给人类带来的启示,和在他完成救赎工程以后,诸宗徒向万民所宣布的得救喜讯。

这喜讯的传报,最初只靠口头的宣讲,稍后才有不少人士把\uline{耶稣}的生平与宣讲笔之于书,因而产生了“福音”的著作。按\uwave{路}1:1的记载,这样的著作在当时已为数不少,可是圣教会自初只承认《玛窦》、《马尔谷》、《路加》、《若望》这四部《福音》为受默感而写的经典,并著录在正经书目内,其他名为“福音”的著作,概著录为伪经。

“福音”书虽有四部,但所传述的“福音”却只是一个,因为四圣史所撰述的是同一的喜讯,只是在所采用形式上有所不同而已。前三部《福音》,无论是在取材和结构上,或在用字上,大致可说相同,甚至可并列对照,一望而知彼此间所有的关系,因而有“对观福音”之称。这三部《福音》之所以如此相同,是因为前三圣史记述了大体相同的“宗徒教理讲授”:\uline{玛窦}记述了\uline{耶路撒冷}教会的传授,\uline{马尔谷}记述了\uline{罗马}教会的传授,\uline{路加}记述了\uline{安提约基}\uline{雅}教会的传授。\uline{若望}因见前三《福音》已流传于世,没有重述的必要,遂由自己记忆所及,采取了一些有关的材料,在第一世纪末叶,针对当时人事环境的需要,编著了自己的《福音》,其目的是在攻击方兴的异端邪说。

四《福音》虽然不是狭义的史书,但就信实性来说:世界上没有一部史书可与之相比,因为各位作者,或是目睹所述之事的宗徒(\uline{玛窦}、\uline{若望}),或是宗徒的亲传弟子(\uline{马尔谷}、\uline{路加}),他们所依据的,全是亲历其事人物的口述;况且《福音》成书时,尚有不少耳闻目睹的证人生存于世。

四《福音》内不但包含了有关信仰绝对重要的道理,而且也给世人提示了诸德的完美模范,基督徒成全的最高理想:即为我们降生成人的天主圣子。