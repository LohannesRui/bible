% 标题
\chapter*{伯多禄后书}

% 右页眉
\rhead{伯多禄后书(伯后)}

% 正文
\textbf{第一章\quad致候辞\quad}
\textsuperscript{1}
\uline{耶稣}\uline{基督}的仆人和宗徒\uline{西满}\uline{伯多禄},致书给那些因我们的天主和救主\uline{耶稣}\uline{基督}的正义,与我们分享同样宝贵信德的人。
\textsuperscript{2}
愿恩宠与平安,因认识天主和我们的主\uline{耶稣},丰富地赐予你们。
\footnote{信德是天主因\uline{耶稣}的“正义”,即因\uline{耶稣}施救的仁慈,赐与人的最宝贵的恩惠(\uwave{罗}1:17并注)。}

\begin{center}
	\textbf{\large{\songti 劝修德行}}
\end{center}

\textbf{对圣召的义务\quad}
\textsuperscript{3}
因为我们认识了那藉自己的光荣和德能,召叫我们的\uline{基督},\uline{基督}天主性的大能,就将各种关乎生命和虔敬的恩惠,赏给了我们,
\textsuperscript{4}
并藉着自己的光荣和德能,将最大和宝贵的恩许赏给了我们,为使你们藉着这些恩许,在逃脱世界上所有败坏的贪欲之后,能成为有分于天主性体的人。
\textsuperscript{5}
正为了这个原故,你们要全力奋勉,在你们的信仰上还要加毅力,在毅力上加知识,
\textsuperscript{6}
在知识上加节制,在节制上加忍耐,在忍耐上加虔敬,
\textsuperscript{7}
在虔敬上加兄弟的友爱,在兄弟的友爱上加爱德。
\textsuperscript{8}
实在,这些德行如果存在你们内,且不断增添,你们决不致于在认识我们的主\uline{耶稣}\uline{基督}上,成为不工作,不结果实的人,
\textsuperscript{9}
因为那没有这些德行的,便是瞎子,是近视眼,忘却了他从前的罪恶已被清除。
\textsuperscript{10}
为此,弟兄们,你们更要尽心竭力,使你们的蒙召和被选,赖善行而坚定不移;倘若你们这样作,决不会跌倒。
\textsuperscript{11}
的确,这样你们便更有把握,进入我们的主和救主\uline{耶稣}\uline{基督}永远的国。
\footnote{信友因认识\uline{耶稣},获得的最大恩惠是宠爱;赖此宠爱,人可分享天主的性体,即是说:在今世作天主的子女(\uwave{罗}8:14;\uwave{若}一1:1),在来世分享天主的光荣(\uwave{若}一1:2)。人既获得如此的尊位,就当度圣善的生活,勉修各种美德,才不愧作天主的子女。}

\textbf{知死期已近\quad}
\textsuperscript{12}
为此,纵然你们已知道这些事,已坚定在所怀有的真理之上,我还是要常提醒你们。
\textsuperscript{13}
我以为只要我还在这帐幕内,就有义务以劝言来鼓励你们。
\textsuperscript{14}
我知道我的帐幕快要拆卸了,一如我们的主\uline{耶稣}\uline{基督}指示给我的。
\textsuperscript{15}
我要尽心竭力使你们在我去世以后,也时常纪念这些事。
\footnote{“帐幕”是指肉躯(\uwave{格}后5:4):言人生短暂,有如原无定所而随处可搭建的帐幕。由此处可知:本书应视为宗徒之长的遗嘱。14节见\uwave{若}21:18、19。}

\textbf{基督再来的确证\quad}
\textsuperscript{16}
我们将我们的主\uline{耶稣}\uline{基督}的大能和来临,宣告给你们,并不是依据虚构的荒诞故事,而是因为我们亲眼见过他的威荣。
\textsuperscript{17}
他实在由天主接受了尊敬和光荣,因那时曾有这样的声音,从显赫的光荣中发出来,向他说:“这是我的爱子,我所喜悦的。”
\textsuperscript{18}
这来自天上的声音,是我们同他在那座圣山上的时候,亲自听见的。
\textsuperscript{19}
因此,我们认定先知的话更为确实,对这话你们当十分留神,就如留神在暗中发光的灯,直到天亮,晨星在你们的心中升起的时候。
\textsuperscript{20}
最主要的,你们应知道经上的一切预言,决不应随私人的解释,
\textsuperscript{21}
因为预言从来不是由人的意愿而发的,而是由天主所派遣的圣人,在圣神推动之下说出来的。
\footnote{\uline{伯多禄}为使信徒切望主的光荣来临,遂提出他在\uline{耶稣}显圣容时的所见所闻(\uwave{玛}17:1-9),以证明众先知有关默西亚光荣来临的预言,必要应验。信友在今世好像在黑暗中生活,但是先知的预言,就如“发光的灯”已在黑暗中照耀,总有一天“晨星”的光,即\uline{耶稣}显现的光耀,要把今世的黑暗完全驱散。20、21两节说明圣经上所记载的一切,是因圣神的默感而写的,因此为明白圣经的话,信友应信赖圣神的光照,以及他为教训万民所用的圣教会的训导(\uwave{弟}后3:14-17;\uwave{弟}前3:15)。}

\begin{center}
	\textbf{\large{\songti 驳斥假教师}}
\end{center}

\textbf{第二章\quad假教师必要出现\quad}
\textsuperscript{1}
从前连在选民中,也有过假先知;同样,将来在你们中,也要出现假教师,他们要倡导使人丧亡的异端,连救赎他们的主,也都敢否认:这是自取迅速的丧亡。
\textsuperscript{2}
有许多人将要随从他们的放荡,甚至真理之道,也要因他们而受到诽谤。
\textsuperscript{3}
他们因贪吝成性,要以花言巧语在你们身上营利;可是他们的案件自古以来,就未安闲,他们的丧亡也决不稍息。
\footnote{就如在旧约中出过假先知(\uwave{列}上22:5-12;\uwave{耶}23:9-32),同样在新约时代,也将有许多讲异端的“假教师”。此处所说的异端,主要的错误是否认\uline{基督}为主(1、10两节),及否认\uline{基督}再来(3:4),误解福音的自由(19节),因而生活放荡,贪吝成性(2、3、12-16节)。}

\textbf{天主惩罚恶人的例证\quad}
\textsuperscript{4}
天主既然没有宽免犯罪的天使,把他们投入了地狱,囚在幽暗的深坑,拘留到审判之时;
\textsuperscript{5}
既然没有宽免古时的世界,曾引来洪水淹灭了恶人的世界,只保存了宣讲正义的\uline{诺厄}一家八口;
\textsuperscript{6}
又降罚了\uline{索多玛}和\uline{哈摩辣}城,使之化为灰烬,至于毁灭,以作后世作恶者的鉴戒,
\textsuperscript{7}
只救出了那因不法之徒的放荡生活而悲伤的义人\uline{罗特}——
\textsuperscript{8}
因为这义人住在他们中,他正直的灵魂,天天因所见所闻的不法行为,感到苦恼——
\textsuperscript{9}
那么,上主自然也知道拯救虔诚的人,脱离磨难,而存留不义的人,等候审判的日子受处罚,
\textsuperscript{10}
尤其是存留那些随从肉欲,而生活在污秽情欲中的人,以及那些轻视“主权者”的人。
\footnote{作者为证明传布异端的假教师的丧亡是必然的,遂征引《旧约》中天主罚恶人的例子为证(\uwave{创}6-8章,19:23-29);且按\uline{犹太}人的传授,征引了创世之前,天使背命受罚的事(\uwave{默}12:7-9;\uwave{若}8:44)。10节“主权者”是指\uline{基督}。}

\textbf{假教师的素描\quad}
他们都是些胆大骄傲的人,竟不怕亵渎“众尊荣者”,
\textsuperscript{11}
就是连力量德能大于他们的天使,也不敢在上主面前,以侮辱的言词对他们下判决。
\textsuperscript{12}
然而这些人实在如无理性的牲畜,生来就是为受捉拿,受宰杀,凡他们不明白的事就要亵渎;他们必要如牲畜一样丧亡,
\textsuperscript{13}
受他们不义的报应。他们只以一日的享受为快乐,实是些污秽肮脏的人;当他们同你们宴乐时,纵情于淫乐;
\textsuperscript{14}
他们满眼邪色,犯罪不餍,勾引意志薄弱的人;他们的心习惯了贪吝,真是些应受咒骂的人。
\textsuperscript{15}
他们离奇正道,走入了歧途,随从了\uline{贝敖尔}的儿子\uline{巴郎}的道路,他曾贪爱过不义的酬报,
\textsuperscript{16}
可是也受了他作恶的责罚:一个不会说话的牲口,竟用人的声音说了话,制止了这先知的妄为。
\textsuperscript{17}
他们像无水的泉源,又像为狂风所飘扬的云雾:为他们所存留的,是黑暗的幽冥。
\textsuperscript{18}
因为他们好讲虚伪的大话,用肉欲的放荡为饵,勾引那些刚才摆脱错谬生活的人;
\textsuperscript{19}
应许他们自由,自己却是败坏的奴隶,因为人被谁制胜,就是谁的奴隶。
\footnote{这般假教师的最大毛病,除狂妄自大外,即是贪财好色。《旧约》里的\uline{巴郎}术士,即是因这两大罪恶而知名的。事见\uwave{户}22-24,25:18,31:16;因此\uline{巴郎}在\uline{犹太}人看来,是贪吝和邪淫的象征人物(\uwave{默}2:14),同时也是罪人被罚的预像。11节“众尊荣者”与\uwave{犹}9节对照,当是指被罚的天使。}

\textbf{背信者的不幸\quad}
\textsuperscript{20}
如果他们因认识主和救世者\uline{耶稣}\uline{基督},而摆脱世俗的污秽以后,再为这些事所缠绕而打败,他们末后的处境,就必比以前的更为恶劣,
\textsuperscript{21}
因为不认识正义之道,比认识后而又背弃那传授给他们的圣诫命,为他们倒好得多。
\textsuperscript{22}
在他们身上正应验了这句恰当的俗语:“狗呕吐的,它又回来再吃”;又“母猪洗净了,又到污泥里打滚”。
\footnote{关于背教者的不幸,见\uwave{玛}12:45;\uwave{希}6:1-8,10:26-39。22节引用的俗语,实在说尽了背弃信德者如何可怜和愚蠢(\uwave{箴}26:11)。}

\textbf{第三章\quad应坚持信仰\quad}
\textsuperscript{1}
亲爱的诸位!这已是我给你们写的第二封信,在这两封信中,我都用提醒的话,来鼓励你们应有赤诚的心,
\textsuperscript{2}
叫你们想起圣先知们以前说过的话,以及你们的宗徒们传授的主和救世者的诫命。
\textsuperscript{3}
首先你们该知道:在末日要出现一些爱嘲笑戏弄,按照自己的私欲生活的人,
\textsuperscript{4}
他们说:“哪里有他所应许的来临?因为自从我们的父老长眠以来,一切照旧存在,全如创造之初一样。”
\footnote{假教师的生活所以如此放荡,是因为他们不信\uline{基督}再次来临,不信世界穷尽,甚至嘲笑\uline{基督}所应许的再次来临。他们的理由是:第一代教友(\uwave{玛}24:34、35)都先后去了世,但还不见\uline{基督}再来。参阅\uwave{得}后2:1-8。}

\textbf{主的日子必要来临\quad}
\textsuperscript{5}
他们故意忘记了:在太古之时,因天主的话,就有了天,也有了由水中出现,并在水中而存在的地;
\textsuperscript{6}
又因天主的话和水,当时的世界为水所淹没而消失了;
\textsuperscript{7}
甚至连现有的天地,还是因天主的话得以保存,直存到审判及恶人丧亡的日子,被火焚烧。

\textsuperscript{8}
亲爱的诸位,惟有这一件事你们不可忘记:就是在天主前一日如千年,千年如一日。
\textsuperscript{9}
主决不迟延他的应许,有如某些人所想象的;其实是他对你们含忍,不愿任何人丧亡,只愿众人回心转意。
\textsuperscript{10}
可是,主的日子必要如盗贼一样到来;在那一日,天要轰然过去,所有的原质都要因烈火而熔化,大地及其中所有的工程,也都要被焚毁。
\footnote{本段即是作者针对假教师的异端,一一加以驳斥:首先说明现有的世界已不完全是原先的世界,已经过一次洪水的淹没(\uwave{创}6-8)。天主既能以一言创造了世界(\uwave{创}1:6-10),又以一言命洪水消灭了罪恶满盈的世界,那么将来也可以一言消灭现有的世界(5-7节)。至于\uline{基督}迟迟不来,一方面是由于时间在天主前与人前有所不同(8节;\uwave{咏}90:4);另一方面是天主不愿罪人丧亡,等待他们悔改。当天主预定的末日一到,天地要顷刻之间化为乌有,一切要为“火”所焚毁(\uwave{玛}24:35;\uwave{格}前3:13)。}

\textbf{应以善生等候主的来临\quad}
\textsuperscript{11}
这一切既然都要这样消失,那么,你们应该怎样以圣洁和虔敬的态度生活,
\textsuperscript{12}
以等候并催促天主日子的来临!在这日子上,天要为火所焚毁,所有的原质也要因烈火而熔化;
\textsuperscript{13}
可是,我们却按照他的应许,等候正义常住在其中的新天新地。
\textsuperscript{14}
为此,亲爱的诸位,你们既然等候这一切,就应该勉力,使他见到你们没有玷污,没有瑕疵,安然无惧;
\textsuperscript{15}
并应以我们主的容忍当作得救的机会;这也是我们可爱的弟兄\uline{保禄},本着赐与他的智慧,曾给你们写过的;
\textsuperscript{16}
也正如他在谈论这些事时,在一切书信内所写过的。在这些书信内,有些难懂的地方,不学无术和站立不稳的人,便加以曲解,一如曲解其他经典一样,而自趋丧亡。
\footnote{世界既有毁灭的一日,信友便不该贪恋世上的财物,应该以虔诚度圣善的生活,期待唯有“正义常住在其中的新天新地”(\uwave{默}21:1-5、27)。作者为强调自己的话,便引用\uline{保禄}的书信为证。由此可知\uline{保禄}的书信,其时已被认为圣经,已与其他经典等量齐观(15、16两节)。\uline{伯多禄}在此处且已暗示《新约经书》的默感性与正经性,并立定了解经的规律,以警戒那些自由解经的人(1:20、21)。}

\textbf{最后劝勉与祝福\quad}
\textsuperscript{17}
所以,亲爱的诸位,你们既预先知道了这些事,就应该提防,免得为不法之徒的错谬所诱惑,而由自己的坚固立场跌下来。
\textsuperscript{18}
你们却要在恩宠及认识我们的主,和救世者\uline{耶稣}\uline{基督}上渐渐增长。愿光荣归于他,从如今直到永远之日,阿门。