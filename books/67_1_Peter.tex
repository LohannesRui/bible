% 标题
\chapter*{伯多禄前书}

% 右页眉
\rhead{伯多禄前书(伯前)}

% 正文
\textbf{第一章\quad致候辞\quad}
\textsuperscript{1}
\uline{耶稣}\uline{基督}的宗徒\uline{伯多禄}致书给散居在\uline{本都}、\uline{迦拉达}、\uline{卡帕多细雅}、\uline{亚细亚}和\uline{彼提尼雅}作旅客的选民:
\textsuperscript{2}
你们被召选,是照天主的预定;受圣神祝圣,是为服事\uline{耶稣}\uline{基督},和分沾他宝血洗净之恩。

愿恩宠和平安丰富地赐予你们!
\footnote{基督徒在世上好似“旅客”,因为他们的本乡是天堂(2:11;\uwave{格}后5:1、6;\uwave{斐}3:20)。“散居……蒙选者”是指散居在外教人中的信友。2节中一一提出天主圣三在救赎工程上每位的工作,以及每位对信友的关系。}

\textbf{序言\quad得沾救恩的福分\quad}
\textsuperscript{3}
愿我们的主\uline{耶稣}\uline{基督}的天父和父受赞美!他因自己的大仁慈,藉\uline{耶稣}\uline{基督}由死者中的复活,重生了我们,为获得那充满生命的希望,
\textsuperscript{4}
为获得那为你们已存留在天上的不坏、无瑕、不朽的产业,
\textsuperscript{5}
因为你们原是为天主的能力所保护,为使你们藉着信德,而获得那已准备好,在最后时期出现的救恩。
\textsuperscript{6}
为此,你们要欢跃,虽然如今你们暂时还该在各种试探中受苦,
\textsuperscript{7}
这是为使你们的信德,得以精炼,比经过火炼而仍易消失的黄金,更有价值,好在\uline{耶稣}\uline{基督}显现时,堪受称赞、光荣和尊敬。
\textsuperscript{8}
你们虽然没有见过他,却爱慕他;虽然你们如今仍看不见他,还是相信他;并且以不可言传,和充满光荣的喜乐而欢跃,
\textsuperscript{9}
因为你们已把握住信仰的效果:灵魂的救恩。
\textsuperscript{10}
关于这救恩,那些预言了你们要得恩宠的先知们,也曾经寻求过,考究过,
\textsuperscript{11}
就是考究那在他们内的\uline{基督}的圣神,预言那要临于\uline{基督}的苦难,和以后的光荣时,指的是什么时期,或怎样的光景。
\textsuperscript{12}
这一切给他们启示出来,并不是为他们自己,而是为给你们服务;这一切,如今藉着给你们宣传福音的人,依赖由天上派遣来的圣神,传报给你们;对于这一切奥迹,连众天使也都切望窥探。
\footnote{作者以赞颂天主圣三对信友所赐的救恩,而反映信友对这救恩所有的义务:信友既信天主给自己准备了永存不朽的产业(救恩),就应怀着喜乐的心情承受天主的试探,仰望来日的光荣。古先知所切望的救恩,连天使也愿知道的奥迹,都启示给信仰\uline{基督}的人了(\uwave{路}24:26、27;\uwave{玛}13:16、17;\uwave{格}前2:9-13;\uwave{哥}1:16)。12节论天使对救恩的奥迹所得的知识,见\uwave{弗}3:10。}

\begin{center}
	\textbf{\large{\songti 信友对天主的义务}}
\end{center}

\textbf{应度圣洁的生活\quad}
\textsuperscript{13}
为此,你们要束上腰,谨守心神,要清醒,要全心希望在\uline{耶稣}\uline{基督}显现时,给你们带来的恩宠;
\textsuperscript{14}
要做顺命的子女,不要符合你们昔日在无知中生活的欲望,
\textsuperscript{15}
但要像那召叫你们的圣者一样,在一切生活上是圣的,
\textsuperscript{16}
因为经上记载:“你们应是圣的,因为我是圣的。”
\textsuperscript{17}
你们既称呼那不看情面,而只按每人的作为行审判者为父,就该怀着敬畏,度过你们这旅居的时期。
\textsuperscript{18}
该知道:你们不是用能朽坏的金银等物,由你们祖传的虚妄生活中被赎出来的,
\textsuperscript{19}
而是用宝血,即无玷无瑕的羔羊\uline{基督}的宝血。
\textsuperscript{20}
他固然是在创世以前就被预定了的,但在最末的时期为了你们才出现,
\textsuperscript{21}
为使你们因着他,而相信那使他由死者中复活,并赐给他光荣的天主:这样你们的信德和望德,都同归于天主。
\footnote{信友既怀着永生的希望(3节),那么,一生的行动生活,便该持守圣洁:因为,一、召选他们为其子女的天主是圣的,他愿他所爱的子女也是圣洁的(\uwave{肋}11:44、45;\uwave{玛}5:48);二、天父不只是圣洁的父,而也是按公义行审判的父(\uwave{罗}2:11);三、信友自身的赎价,并不是金银财宝,而是\uline{基督}的宝血(\uwave{希}9:14;\uwave{默}5:6-13)。}

\textbf{赤诚相爱\quad}
\textsuperscript{22}
你们既因服从真理,而洁净了你们的心灵,获得了真实无伪的弟兄之爱,就该以赤诚的心,热切相爱,
\textsuperscript{23}
因为你们原是赖天主生活而永存的圣言,不是由于能坏的,而是由于不能坏的种子,得以重生。
\textsuperscript{24}
因为“凡有血肉的都似草,他的一切美丽都似草上的花:草枯萎了,花也就凋谢了;
\textsuperscript{25}
但上主的话却永远常存”:这话就是传报给你们的福音。
\footnote{24、25两节引自\uwave{依}40:6、8,证明信友由之而生的种子——天主的话——福音(见\uwave{雅}1:18),是永存不朽的。}

\textbf{第二章\quad应以基督为基石\quad}
\textsuperscript{1}
所以你们应放弃各种邪恶、各种欺诈、虚伪、嫉妒和各种诽谤,
\textsuperscript{2}
应如初生的婴儿贪求属灵性的纯奶,为使你们靠着它生长,以致得救;
\textsuperscript{3}
何况你们已尝到了“主是何等的甘饴”。
\textsuperscript{4}
你们接近了他,即接近了那为人所摈弃,但为天主所精选,所尊重的活石,
\textsuperscript{5}
你们也就成了活石,建成一座属神的殿宇,成为一班圣洁的司祭,以奉献因\uline{耶稣}\uline{基督}而中悦天主的属神的祭品。
\textsuperscript{6}
这就是经上所记载的:“看,我要在\uline{熙雍}安放一块精选的,宝贵的基石,凡信赖他的,决不会蒙羞。”
\textsuperscript{7}
所以为你们信赖的人,是一种荣幸;但为不信赖的人,是“匠人弃而不用的石头,反而成了屋角的基石”;
\textsuperscript{8}
并且是“一块绊脚石,和一块使人跌倒的磐石”。他们由于不相信天主的话,而绊倒了,这也是为他们预定了的。
\textsuperscript{9}
至于你们,你们却是特选的种族,王家的司祭,圣洁的国民,属于主的民族,为叫你们宣扬那由黑暗中召叫你们,进入他奇妙之光者的荣耀。
\textsuperscript{10}
你们从前不是天主的人民,如今却是天主的人民;从前没有蒙受爱怜,如今却蒙受了爱怜。
\footnote{信友过的圣洁生活,只有与\uline{基督}结合才能增进发展。原来信友之与\uline{基督},有如婴儿之与母亲(2、3两节),建筑物之与“基石”(4、5两节),因为信友不能脱离\uline{基督}而独存;何况信友因\uline{基督}才能成为“活石”,与\uline{基督}一起建成一座属于天主的圣殿(\uwave{格}前3:16、17;\uwave{弗}2:20-22),与\uline{基督}共同形成一司祭团(\uwave{默}1:1,5:10),将自己圣洁的生活,作为馨香的祭品,藉\uline{基督}奉献于天主(\uwave{罗}12:1;\uwave{斐}2:17)。3节引自\uwave{咏}34:9。6-8节,见\uwave{依}28:16,8:14;\uwave{咏}118:22。}

\textbf{应服从政权\quad}
\textsuperscript{11}
亲爱的!我劝你们作侨民和作旅客的,应戒绝与灵魂作战的肉欲;
\textsuperscript{12}
在外教人中要常保持良好的品行,好使那些诽谤你们为作恶者的人,因见到你们的善行,而在主眷顾的日子,归光荣于天主。
\textsuperscript{13}
你们要为主的缘故,服从人立的一切制度:或是服从帝王为最高的元首,
\textsuperscript{14}
或是服从帝王派遣来惩罚作恶者,奖赏行善者的总督,
\textsuperscript{15}
因为这原是天主的旨意:要你们行善,使那些愚蒙无知的人,闭口无言。
\textsuperscript{16}
你们要做自由的人,却不可做以自由为掩饰邪恶的人,但该做天主的仆人;
\textsuperscript{17}
要尊敬众人,友爱弟兄,敬畏天主,尊敬君王。
\footnote{信友应乐意服从一切合法的政府,因为一切权柄都是来自天主(\uwave{罗}13:1-7)。惟有在政府滥用政权时,人民才没有服从的义务(\uwave{宗}4:19、20,5:29)。信友如奉公守法(13、14两节),敬主爱人(17节),就可使那些侮蔑信友的人闭口无言。天主甚或有时以信友的善表感动(“眷顾”12节)他们,使他们归依圣教。}

\textbf{仆人对主人应有的态度\quad}
\textsuperscript{18}
你们做家仆的,要以完全敬畏的心服从主人,不但对良善和温柔的,就是对残暴的,也该如此。
\textsuperscript{19}
谁若明知是天主的旨意,而忍受不义的痛苦:这才是中悦天主的事。
\textsuperscript{20}
若你们因犯罪被打而受苦,那还有什么光荣?但若因行善而受苦,而坚心忍耐:这才是中悦天主的事。
\textsuperscript{21}
你们原是为此而蒙召的,因为\uline{基督}也为你们受了苦,给你们留下了榜样,叫你们追随他的足迹。
\textsuperscript{22}
“他没有犯过罪,他口中也从未出过谎言”;
\textsuperscript{23}
他受辱骂,却不还骂;他受虐待,却不报复,只将自己交给那照正义行审判的天主;
\textsuperscript{24}
他在自己的身上,亲自承担了我们的罪过,上了木架,为叫我们死于罪恶,而活于正义;“你们是因他的创伤而获得了痊愈”。
\textsuperscript{25}
你们从前有如迷途的亡羊,如今却被领回,归依你们的灵牧和监督。
\footnote{为奴的信友应如天主的仆人一样,服从主人。有关奴隶的劝言,见\uwave{费}、\uwave{弗}6:5-8;\uwave{哥}3:22-25等处。作者为鼓励为奴的信友,情愿为主忍受无理的虐待,特以\uline{基督}的忍耐作榜样。22-25节,见\uwave{依}53。}

\textbf{第三章\quad夫妇间应遵守的义务\quad}
\textsuperscript{1}
同样,你们做妻子的,应当服从自己的丈夫,好叫那些不信从天主话的,为了妻子无言的品行而受感化,
\textsuperscript{2}
因为他们看见了,你们怀有敬畏的贞洁品行。
\textsuperscript{3}
你们的装饰不应是外面的发型、金饰,或衣服的装束,
\textsuperscript{4}
而应是那藏于内心,基于不朽的温柔,和宁静心神的人格:这在天主前才是宝贵的。
\textsuperscript{5}
从前那些仰望天主的圣妇,正是这样装饰了自己,服从了自己的丈夫。
\textsuperscript{6}
就如\uline{撒辣}听从了\uline{亚巴郎},称他为“主”;你们如果行善,不害怕任何恐吓,你们就是她的女儿。

\textsuperscript{7}
同样,你们作丈夫的,应该凭着信仰的智慧与妻子同居,待她们有如较为脆弱的器皿,尊敬她们,有如与你们共享生命恩宠的继承人:这样你们的祈祷便不会受到阻碍。
\footnote{关于夫妇间彼此应遵守的义务,见\uwave{格}前7;\uwave{弗}5:22-33;\uwave{哥}3:18、19等处。做妻子的基本美德是服从;做丈夫的基本美德是爱怜。有了这两种基本美德,家庭的生活必能和谐幸福。6节\uline{撒辣}为信徒之母,见\uwave{迦}4:22-30;\uwave{罗}9:8、9。}

\textbf{应以爱德与人相处\quad}
\textsuperscript{8}
总之,你们都该同心合意,互表同情,友爱弟兄,慈悲为怀,谦逊温和;
\textsuperscript{9}
总不要以恶报恶,以骂还骂;但要祝福,因为你们原是为继承祝福而蒙召的。
\textsuperscript{10}
所以“凡愿意爱惜生命,和愿意享见幸福日子的,就应谨守口舌,不说坏话,克制嘴唇,不言欺诈;
\textsuperscript{11}
躲避邪恶,努力行善,寻求和平,全心追随,
\textsuperscript{12}
因为上主的双目垂顾正义的人,他的两耳俯听他们的哀声;但上主的威容敌视作恶的人。”
\footnote{9节见\uwave{玛}5:38-48;\uwave{路}6:28。10-12节引自\uwave{咏}34:13-17。}

\begin{center}
	\textbf{\large{\songti 苦难中应有的态度}}
\end{center}

\textbf{应安心受苦\quad}
\textsuperscript{13}
如果你们热心行善,谁能加害你们呢?
\textsuperscript{14}
但若你们为正义而受苦,才是有福的。你们不要害怕人们的恐吓,也不要心乱。
\textsuperscript{15}
你们但要在心内尊崇\uline{基督}为主;若有人询问你们心中所怀希望的理由,你们要时常准备答复,
\textsuperscript{16}
且要以温和、以敬畏之心答复,保持纯洁的良心,好使那些诬告你们在\uline{基督}内有良好品行的人,在他们诽谤你们的事上,感到羞愧。
\footnote{14、15两节见\uwave{玛}5:10;\uwave{依}8:12、13。}

\textbf{基督无辜受苦的模范\quad}
\textsuperscript{17}
若天主的旨意要你们因行善而受苦,自然比作恶而受苦更好,
\textsuperscript{18}
因为\uline{基督}也曾一次为罪而死,且是义人代替不义的人,为将我们领到天主面前;就肉身说,他固然被处死了;但就神魂说,他却复活了。
\footnote{见\uwave{希}9:26-28。所谓“神魂”是指\uline{耶稣}的灵魂,因为他的灵魂赖与她结合的天主性,有使他的肉身复活的能力。}
\textsuperscript{19}
他藉这神魂,曾去给那些在狱中的灵魂宣讲过;
\textsuperscript{20}
这些灵魂从前在\uline{诺厄}建造方舟的时日,天主耐心期待之时,原是不信的人;当时赖方舟经过水而得救的不多,只有八个生灵。
\footnote{\uline{耶稣}死后,他的灵魂给“在狱中的灵魂”,即给一切集中在“灵薄狱”的古善人的灵魂,报告了救恩的喜讯(4:6)。信经中“我信其降地狱”一句,即含有此意。作者在此另外提出被洪水淹没而在死前回头的人,为证明救恩的效力。关于洪水,见\uwave{创}6-9章。}
\textsuperscript{21}
这水所预表的圣洗,如今赖\uline{耶稣}\uline{基督}的复活拯救了你们,并不是涤除肉体的污秽,而是向天主要求一纯洁的良心。
\footnote{就如洪水浮起\uline{诺厄}的方舟,救了“八个生灵”(\uwave{创}7:7),这样圣洗圣事使信友得免于永远的沦亡。}
\textsuperscript{22}
至于\uline{耶稣}\uline{基督},他升了天,坐在天主的右边,
\footnote{见\uwave{弗}1:20-22。}
众天使、掌权者和异能者都屈伏在他权下。

\textbf{第四章\quad应度圣洁的生活\quad}
\textsuperscript{1}
\uline{基督}既然在肉身上受了苦难,你们就应该具备同样的见识,深信凡在肉身上受苦的,便与罪恶断绝了关系;
\textsuperscript{2}
今后不再顺从人性的情欲,而只随从天主的意愿,在肉身内度其余的时日。
\textsuperscript{3}
过去的时候,你们实行外教人的欲望,生活在放荡、情欲、酗酒、宴乐、狂饮和违法的偶像崇拜中,这已经够了!
\footnote{信友既因\uline{基督}的苦难圣死,藉圣洗已死于罪恶,活于天主,便常该纪念\uline{基督}的苦难(1节),怀有他“所怀有的心情”(\uwave{斐}2:5),远避罪恶(2、3两节)。}
\textsuperscript{4}
由于你们不再同他们狂奔于淫荡的洪流中,他们便引以为怪,遂诽谤你们;
\textsuperscript{5}
但他们要向那已准备审判生死者的主交账。
\textsuperscript{6}
也正是为此,给死者宣讲了这福音:他们虽然肉身方面如同人一样受了惩罚,可是神魂方面却同天主一起生活。
\footnote{“给死者宣讲了福音”一语,似乎应照3:18-20来解释(见3注五)。因为人死后,他永远祸福的结局已成定案,再无悔改的可能(\uwave{伯}后2:4;\uwave{玛}25:41;\uwave{路}16:25-31)。被洪水淹死的人,照人的看法是受了天主的严罚,但是天主却看见其中有许多因悔罪而得救的人。}

\textsuperscript{7}
万事的结局已临近了,所以你们应该慎重,应该醒寤祈祷。
\textsuperscript{8}
最重要的是:你们应该彼此热切相爱,因为爱德遮盖许多罪过;
\textsuperscript{9}
要彼此款待,而不出怨言。
\textsuperscript{10}
各人应依照自己所领受的神恩,彼此服事,善做天主各种恩宠的管理员。
\textsuperscript{11}
谁若讲道,就该按天主的话讲;谁若服事,就该本着天主所赐的德能服事,好叫天主在一切事上,因\uline{耶稣}\uline{基督}而受到光荣:愿光荣和权能归于他,至于无穷之世。阿门。
\footnote{8节见\uwave{雅}5:20。10、11两节:运用天主所赐的特恩,应依照天主的旨意,为光荣\uline{基督},切不可图谋私益和虚荣(\uwave{格}前12章)。}

\textbf{应乐于受苦\quad}
\textsuperscript{12}
亲爱的,你们不要因为在你们中,有试探你们的烈火而惊异,好像遭遇了一件新奇的事;
\textsuperscript{13}
反而要喜欢,因为分受了\uline{基督}的苦难,这样好使你们在他光荣显现的时候,也能欢喜踊跃。
\textsuperscript{14}
如果你们为了\uline{基督}的名字,受人辱骂,便是有福的,因为光荣的神,即天主的神,就安息在你们身上。
\textsuperscript{15}
惟愿你们中谁也不要因做凶手,或强盗,或坏人,或做煽乱的人而受苦。
\textsuperscript{16}
但若因为是基督徒而受苦,就不该以此为耻,反要为这名称光荣天主,
\footnote{宗徒劝告新奉教的人,蒙召入\uline{基督}所立的教会,在世上难免不遭受迫害;并且无辜为\uline{基督}而受人迫害,正是蒙拣选的表记,所以是可庆幸的事(\uwave{玛}5:11、12)。}
\textsuperscript{17}
因为时候已经到了,审判必从天主的家开始;如果先从我们开始,那些不信从天主福音者的结局,又将怎样呢?
\textsuperscript{18}
“如果义人还难以得救,那么恶人和罪人,要有什么结果呢?”
\textsuperscript{19}
故此,凡照天主旨意受苦的人,也要把自己的灵魂托付给忠信的造物主,专务行善。
\footnote{17节“天主的家”指圣教会(\uwave{弟}前3:15)。18节见\uwave{箴}11:31。}

\begin{center}
	\textbf{\large{\songti 各项劝言}}
\end{center}

\textbf{第五章\quad司牧与信友间的义务\quad}
\textsuperscript{1}
所以我这同为长老的,为\uline{基督}苦难作证的,以及同享那将要显示的光荣的人,劝勉你们中间的众长老:
\textsuperscript{2}
你们务要牧放天主托付给你们的羊群;尽监督之职,不是出于不得已,而是出于甘心,随天主的圣意;也不是出于贪卑鄙的利益,而是出于情愿;
\textsuperscript{3}
不是做托你们照管者的主宰,而是做群羊的模范:
\textsuperscript{4}
这样,当总司牧出现时,你们必领受那不朽的荣冠。
\footnote{4:17作者曾提及“天主的家”必要先受审判的话,那么那些管理各地教会的长老,既是管理天主的家人,因此同为长老的\uline{伯多禄}就劝告他们善尽己职,做群羊的模范,这样在审判时,他们才可领受光荣的花冠。\uline{耶稣}为“总司牧”见\uwave{希}13:20。}

\textsuperscript{5}
同样,你们青年人,应该服从长老;大家都该穿上谦卑作服装,彼此侍候,因为“天主拒绝骄傲人,却赏赐恩宠于谦逊人”。
\textsuperscript{6}
为此,你们该屈服在天主大能的手下,这样在适当的时候,他必举扬你们;
\textsuperscript{7}
将你们的一切挂虑都托给他,因为他必关照你们。
\footnote{“青年人”是指一切信友。5节见\uwave{雅}4:6;7节见\uwave{咏}55:23。}

\textbf{最后劝言\quad}
\textsuperscript{8}
你们要节制,要醒寤,因为,你们的仇敌魔鬼,如同咆哮的狮子巡游,寻找可吞食的人;
\textsuperscript{9}
应以坚固的信德抵抗他,也该知道:你们在世上的众弟兄,都遭受同样的苦痛。
\textsuperscript{10}
那赐万恩的天主,即在\uline{基督}内召叫你们进入他永远光荣的天主,在你们受少许苦痛之后,必要亲自使你们更为成全、坚定、强健、稳固。
\textsuperscript{11}
愿光荣与权能归于他,至于无穷之世。阿门。
\footnote{基督徒要以坚固的信德抵抗魔鬼(\uwave{雅}4:7),要彼此友爱和互相团结(\uwave{弟}后3:12;\uwave{希}10:24、25),且要对永远的光荣怀有热烈的希望(1:6、7),这样方能忍受一切暂时的苦难,而达于成全的境界。}

\textbf{问安与祝福\quad}
\textsuperscript{12}
我藉忠信的弟兄\uline{息耳瓦诺},给你们写了这封我认为简短的书信,为劝勉你们,并为证明这实在是天主的恩宠:
在这恩宠上你们应该站稳。
\textsuperscript{13}
与你们一同被选的\uline{巴比伦}教会问候你们;我儿\uline{马尔谷}也问候你们。
\textsuperscript{14}
你们要以爱的亲吻,彼此问候。愿平安与在\uline{基督}内的你们众人同在。
\footnote{最后三节可能是\uline{伯多禄}亲手所写,为说明此信是出于他,\uline{息耳瓦诺}(或\uline{息拉},原是\uline{保禄}的伴侣,见\uwave{宗}15:22、40,18:5等处),只是他的代笔人。\uline{巴比伦}暗指\uline{罗马}。关于\uline{马尔谷},见《马尔谷福音》引言。}