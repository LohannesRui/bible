% 标题
\chapter*{迦拉达书}

% 右页眉
\rhead{迦拉达书(迦)}

% 正文
\textbf{第一章\quad致候辞\quad}
\textsuperscript{1}
我\uline{保禄}宗徒——我蒙召为宗徒,并非由于人,也并非藉着人,而是由于\uline{耶稣}\uline{基督}和使他由死者中复活的天主父
\footnote{\uline{保禄}在致候辞内,已将本书的两个主题提出:(一)他作宗徒是直接由天主的召选;(二)\uline{耶稣}的死是人得救的唯一根源。他把这两个主题点出,反击\uline{犹太}保守派对他为宗徒的攻击,并对救恩所有的错误。}
——
\textsuperscript{2}
我和同我在一起的众弟兄,致书给\uline{迦拉达}众教会:
\textsuperscript{3}
愿恩宠与平安由天主我们的父及主\uline{耶稣}\uline{基督}赐与你们!
\textsuperscript{4}
这\uline{基督}按照天主我们父的旨意,为我们的罪恶舍弃了自己,为救我们脱离此邪恶的世代。
\textsuperscript{5}
愿光荣归于天主,至于无穷之世!阿门。

\begin{center}
	\textbf{\large{\songti 福音来自天主}}
\end{center}

\textbf{哀信友动摇之速\quad}
\textsuperscript{6}
我真奇怪,你们竟这样快离开了那以\uline{基督}的恩宠召叫你们的天主,而归向了另一福音;
\textsuperscript{7}
其实,并没有别的福音,只是有一些人扰乱你们,企图改变\uline{基督}的福音而已。
\textsuperscript{8}
但是,无论谁,即使是我们,或是从天上降下的一位天使,若给你们宣讲的福音,与我们给你们所宣讲的福音不同,当受诅咒。
\textsuperscript{9}
我们以前说过,如今我再说:谁若给你们宣讲福音与你们所接受的不同,当受诅咒。
\textsuperscript{10}
那么,我如今是讨人的喜爱,或是讨天主的喜爱呢?难道我是寻求人的欢心吗?如果我还求人的欢心,我就不是\uline{基督}的仆役。
\footnote{使\uline{保禄}深恶痛绝的,是他的教友“离开了”他所讲的福音,而“归向”即投向了\uline{犹太}保守派所讲的“福音”(守\uline{梅瑟}法律),因为使人得救的福音只有一个,决不能因人因时而有所变更。}

\textbf{福音与宗徒职位全由天主而来\quad}
\textsuperscript{11}
弟兄们,我告诉你们;我所宣讲的福音,并不是由人而来的,
\textsuperscript{12}
因为,我不是由人得来的,也不是由人学来的,而是由\uline{耶稣}\uline{基督}的启示得来的。
\textsuperscript{13}
你们一定听说过,我从前尚在\uline{犹太}教中的行动:我怎样激烈地迫害过天主的教会,竭力想把她消灭;
\textsuperscript{14}
我在\uline{犹太}教中比我本族许多同年的人更为急进,对我祖先的传授更富于热枕。
\textsuperscript{15}
但是,从母胎中已选拔我,以恩宠召叫我的天主,却决意
\textsuperscript{16}
将他的圣子启示给我,叫我在异民中传扬他。我当时没有与任何人商量,
\textsuperscript{17}
也没有上\uline{耶路撒冷}去见那些在我以前作宗徒的人,我立即去了\uline{阿剌伯},然后又回到了\uline{大马士革}。
\textsuperscript{18}
此后,过了三年,我才上\uline{耶路撒冷}去拜见\uline{刻法},在他那里逗留了十五天,
\textsuperscript{19}
除了主的兄弟\uline{雅各伯},我没有看见别的宗徒。
\textsuperscript{20}
我给你们写的都是真的,我在天主前作证,我决没有说谎。
\textsuperscript{21}
此后,我往\uline{叙利亚}和\uline{基里基雅}地域去了。
\textsuperscript{22}
那时,\uline{犹太}境内属于\uline{基督}的各教会,都没有见过我的面;
\textsuperscript{23}
只是听说过:“那曾经迫害我们的,如今却传扬他曾经想消灭的信仰了。”
\textsuperscript{24}
他们就为了我而光荣天主。
\footnote{\uline{保禄}所宣讲的福音不是由人传给他的,而是由\uline{耶稣}亲自启示的;他作宗徒也是由于天主的召选(见\uwave{宗}9:1-9)。18节特提出他拜见\uline{伯多禄}的事,表示他承认\uline{伯多禄}在宗徒中的优越地位。}

\textbf{第二章\quad其他宗徒赞同保禄\quad}
\textsuperscript{1}
过了十四年,我同\uline{巴尔纳伯}再上\uline{耶路撒冷}去,还带了\uline{弟铎}同去。
\textsuperscript{2}
我是受了启示而上去的;我在那里向他们陈述了我在异民中间所将的福音,和私下向那些有权威的人陈述过,免得我白白地奔跑,或者徒然奔走了。
\textsuperscript{3}
但是,即连跟我的\uline{弟铎},他虽是\uline{希腊}人,也没有被强迫领受割损,
\textsuperscript{4}
因为,有些潜入的假弟兄,曾要他受割损;这些人潜入了教会,是为窥探我们在\uline{基督}\uline{耶稣}内所享有的自由,好使我们再成为奴隶;
\textsuperscript{5}
可是对他们,我们连片刻时间也没有让步屈服,为使福音的真理在你们中保持不变。
\footnote{“过了十四年”(1节),即\uline{保禄}归化后第十四年(公元49或50年,见\uline{宗}15:2、3)。“再上\uline{耶路撒冷}去”,按1:18他已去过一次;且说“是受了启示而上去的”(2节),可知\uline{保禄}作事常随天主的指示。“那些有权威的”,即9节所称为柱石的三位大宗徒。由“免得我白白地奔跑……”一句,可见他对传福音多么谨慎,决不一意孤行;但对于不合福音真谛,迫使由外邦归化的信友受割损的事(\uwave{宗}15:1),他始终不肯让步。}
\textsuperscript{6}
至于那些所谓有权威的人——不论他们以前是何等人物,与我毫不相干;天主决不顾情面——那些有权威的人,也没有另外吩咐我什么;
\textsuperscript{7}
反而他们看出来,我是受了委托,向未受割损的人,宣传福音,就如\uline{伯多禄}被委派向受割损的人宣传福音一样;
\textsuperscript{8}
因为,那叫\uline{伯多禄}为受割损的人致力尽宗徒之职的,也叫我为外邦人致力尽宗徒之职。
\textsuperscript{9}
所以,他们一认清了所赋与我的恩宠,那称为柱石的\uline{雅各伯}、\uline{刻法}和\uline{若望},就与我和\uline{巴尔纳伯}握手,表示通力合作,叫我们往外邦人那里去,而他们却往受割损的人那里去。
\textsuperscript{10}
他们只要我们怀念穷人;对这一点我也曾尽力行了。
\footnote{这里所提的“穷人”是指在\uline{耶路撒冷}因受迫害而成了穷人的信徒(见\uline{格}前16:1;\uline{格}后8,9)。}

\textbf{安提约基雅的事件\quad}
\textsuperscript{11}
但是,当\uline{刻法}来到\uline{安提约基}\uline{雅}时,我当面反对了他,因为他有可责的地方。
\textsuperscript{12}
原来由\uline{雅各伯}那里来了一些人,在他们未到以前,他惯常同外邦人一起吃饭;可是他们一来到了,他因怕那些受割损的人,就退避了,自己躲开。
\textsuperscript{13}
其余的\uline{犹太}人也都跟他一起装假,以致连\uline{巴尔纳伯}也受了他们的牵引而装假。
\textsuperscript{14}
我一见他们的行为与福音的真理不合,就当着众人对\uline{刻法}说:“你是\uline{犹太}人,竟按照外邦人的方式,而不按照\uline{犹太}人的方式过活,你怎么敢强迫外邦人\uline{犹太}化呢?”
\footnote{1:8及本章2-9各节,充分表现\uline{保禄}决以\uline{伯多禄}为教会的元首,为宗徒之长;正因如此,他的一举一动更能影响信友,因此\uline{保禄}才指摘\uline{伯多禄}。由这样的责难,可见\uline{保禄}的勇敢,\uline{伯多禄}的谦逊。}
\textsuperscript{15}
我们生来是\uline{犹太}人,而不是出于外邦民族的罪人;
\textsuperscript{16}
可是我们知道:人成义不是由于遵行法律,而只是因着对\uline{耶稣}\uline{基督}的信仰,所以我们也信从了\uline{基督}\uline{耶稣},才能由于对\uline{基督}的信仰,而不由于遵行法律成义,因为由于遵守法律,任何人都不得成义。
\textsuperscript{17}
如果我们在\uline{基督}内求成义的人,仍如他们一样被视为罪人,那么,\uline{基督}岂不是成了支持罪恶的人了吗?绝对不是。
\textsuperscript{18}
如果我把我所拆毁的,再修建起来,我就证明我是个罪犯。
\textsuperscript{19}
其实,我已由于法律而死于法律了,为能生活于天主;我已同\uline{基督}被钉在十字架上了,
\textsuperscript{20}
所以,我生活已不是我生活,而是\uline{基督}在我内生活;我现今在肉身内生活,是生活在对天主子的信仰内;他爱了我,且为我舍弃了自己。
\textsuperscript{21}
我决不愿使天主的恩宠无效,因为,如果成义是赖着法律,那么,\uline{基督}就白白地死了。
\footnote{17节的意思是说:如果我们还以为必须守\uline{梅瑟}法律,才可在天主前成义,那么,我们因曾作证只赖\uline{耶稣}的恩宠才可成义得救,就成了罪人,如此,那废弃旧约法律的\uline{耶稣}就应为我们的罪负责。关于19节,参阅\uwave{罗}7注一。}

\begin{center}
	\textbf{\large{\songti 福音使人自由}}
\end{center}

\textbf{第三章\quad成义是由于信德\quad}
\textsuperscript{1}
无知的\uline{迦拉达}人啊!被钉在十字架上的\uline{耶稣}\uline{基督},已活现地摆在你们眼前,谁又迷惑了你们呢?
\textsuperscript{2}
我只愿向你们请教这一点:你们领受了圣神,是由于遵行法律呢?还是由于听信福音呢?
\textsuperscript{3}
你们竟这样无知吗?你们以圣神开始了,如今又愿以肉身结束吗?
\textsuperscript{4}
你们竟白白受了这么多的苦吗?果然是白白地吗?
\textsuperscript{5}
天主赐与你们圣神,并在你们中间施展了德能,是因为你们遵行法律呢?还是因为你们听信福音呢?
\textsuperscript{6}
经上这样记载说:“\uline{亚巴郎}信了天主,天主就以此算为他的正义。”
\textsuperscript{7}
为此你们该晓得:具有信德的人,才是\uline{亚巴郎}的子孙。
\textsuperscript{8}
圣经预见天主将使异民凭信德成义,就向\uline{亚巴郎}预报福音说:“万民都要因你获得祝福。”
\textsuperscript{9}
可见那些具有信德的人,与有信德的\uline{亚巴郎}同蒙祝福。
\footnote{\uline{保禄}在此段论人成义,不是因为遵守\uline{梅瑟}法律,而是因着信德。他先举出\uline{迦拉达}人成义的事实,是由于领受圣神,而不是有赖割损;然后举出\uline{亚巴郎}以信德而非以割损成义为例证。6,8两节所引,见\uline{创}15:6,12:3。}

\textbf{基督废弃了奴隶性的法律\quad}
\textsuperscript{10}
反之,凡是依恃遵行法律的,都应受咒骂,因为经上记载说:“凡不持守《律书》上所记载的一切,而依照遵行的,是可咒骂的。”
\textsuperscript{11}
所以很明显的,没有一个人能凭法律在天主前成义,因为经上说:“义人因信德而生活。”
\textsuperscript{12}
但是法律并非以信德为本,只说:“遵行法令的,必因此获得生命。”
\textsuperscript{13}
但\uline{基督}由法律的咒骂中赎出了我们,为我们成了可咒骂的,因为经上记载说:“凡被悬在木架上的,是可咒骂的。”
\textsuperscript{14}
这样天主使\uline{亚巴郎}所蒙受的祝福,在\uline{基督}\uline{耶稣}内普及于万民,并使我们能藉着信德领受所应许的圣神。
\footnote{\uline{保禄}进一步说明,那不愿以信仰\uline{基督},而只愿以守法律成义的人,使自己成了可咒骂的(参阅\uwave{申}27:26),因为人按自己的力量,不能全守\uline{梅瑟}的条文,所以为成义只应依仗\uline{基督}的恩宠。他为拯救人类,死在十字架上,使自己成了法律所“咒骂的”(参阅\uwave{申21:23}),为使万民因他的死而得到祝福。}
\textsuperscript{15}
弟兄们!就常规来说:连人的遗嘱,如果是正式成立的,谁也不得废除或增订。
\textsuperscript{16}
那么,恩许是向\uline{亚巴郎}和他的后裔所许诺的,并没有说“后裔们”,好像是向许多人说的,而是向一个人,即“你的后裔”,就是指\uline{基督}。
\textsuperscript{17}
我是说:天主先前所正式立定的誓约,决不能为四百三十年以后成立的法律所废除,以致使恩许失效。
\textsuperscript{18}
如果承受产业是由于法律,就已不是由于恩许;但天主是由于恩许把产业赐给了\uline{亚巴郎}。
\footnote{此段(15-18节)证明使人获得救恩的,不是由于法律,而是由于天主向\uline{亚巴郎}起誓所许的恩许;这恩许只赖\uline{基督}而实现了。}

\textsuperscript{19}
那么,为什么还有法律呢?它是为显露过犯而添设的,等他所恩许的后裔来到,它原是藉着天使,经过中人的手而立定的。
\textsuperscript{20}
可是如果出于单方,就不需要中人了,而天主是由单方赐与了恩许。
\textsuperscript{21}
那么,法律相反天主的恩许吗?绝对不是。如果所立定的法律能赐与人生命,正义就的确是出于法律了。
\textsuperscript{22}
但是圣经说过:一切人都被禁锢在罪恶权下,好使恩许藉着对\uline{基督}\uline{耶稣}的信仰,归于相信的人。
\textsuperscript{23}
在“信仰”尚未来到以前,我们都被禁锢在法律的监守之下,以期待“信仰”的出现。
\textsuperscript{24}
这样,法律就成了我们的启蒙师,领我们归于\uline{基督},好使我们由于信仰而成义。
\textsuperscript{25}
但是“信仰”一到,我们就不再处于启蒙师权下了。
\footnote{法律的目的是为维护恩许,是为预防\uline{以色列}人背弃天主,敬拜邪神,沾染外教人的恶习。法律好像启蒙师(护送儿童上学的奴隶),给\uline{以}民准备信仰\uline{默西亚}的道路。19节参阅\uwave{罗}7:7-25。23,25两节中的“信仰”指示新约制度。}
\textsuperscript{26}
其实你们众人都藉着对\uline{基督}\uline{耶稣}的信仰,成了天主的子女,
\textsuperscript{27}
因为你们凡是领了洗归于\uline{基督}的,就是穿上了\uline{基督}:
\textsuperscript{28}
不再分\uline{犹太}人或\uline{希腊}人,奴隶或自由人,男人或女人,因为你们众人在\uline{基督}\uline{耶稣}内已成了一个。
\textsuperscript{29}
如果你们属于\uline{基督},那么,你们就是\uline{亚巴郎}的后裔,就是按照恩许作承继的人。
\footnote{就像成人不再受启蒙师管辖,同样,\uline{犹太}人自从\uline{基督}来了以后,就不再受为启蒙师的法律所约束了。他们既不用遵守旧法律,外邦人更不用遵守,因为人在受洗后,都同样成了天主的子女,都有同样的地位,都穿上\uline{基督}与他密切结合,分享一切恩许。}

\textbf{第四章\quad宠爱使人为天主的子女\quad}
\textsuperscript{1}
再说:承继人几时还是孩童,虽然他是一切家业的主人,却与奴隶没有分别,
\textsuperscript{2}
仍属于监护人和代理人的权下,直到父亲预定的期限。
\textsuperscript{3}
同样,当我们以前还作孩童的时候,我们是隶属于今世的蒙学权下;
\textsuperscript{4}
但时期一满,天主就派遣了自己的儿子来,生于女人,生于法律之下,
\textsuperscript{5}
为把在法律之下的人赎出来,使我们获得义子的地位。
\textsuperscript{6}
为证实你们确实是天主的子女,天主派遣了自己儿子的圣神,到我们心内喊说:“阿爸,父啊!”
\textsuperscript{7}
所以你已不再是奴隶,而是儿子了;如果是儿子,赖天主的恩宠,也成了承继人。
\footnote{作者以未成年的孩童,比作旧约时代的人类,那时无论\uline{犹太}人或外邦人,都好像奴隶,“属于今世蒙学的权下”。所谓“蒙学”是指\uline{犹太}人的法律及教外人对道德观所有的原理,二者都不健全。到了\uline{默西亚}时代,\uline{耶稣}救赎人脱离了法律,提高了人的地位,使人成为天主的子女,人才可呼天主为“阿爸,父啊”(参见\uwave{罗}8:15;\uwave{谷}14:36)。}

\textbf{接受法律是再愿为奴\quad}
\textsuperscript{8}
当你们还不认识天主的时候,服事了一些本来不是神的神;
\textsuperscript{9}
但如今你们认识了天主,更好说为天主所认识;那么,你们怎么又再回到那无能无用的蒙学里去,情愿再作他们的奴隶呢?
\textsuperscript{10}
你们竟又谨守某日、某月、某时、某年!
\textsuperscript{11}
我真为你们担心,怕我白白地为你们辛苦了。
\footnote{\uline{迦拉达}人今若仍遵守\uline{犹太}人的法律,便是自愿放弃自由,再作奴隶。}

\textbf{劝迦拉达人不要受人愚弄\quad}
\textsuperscript{12}
弟兄们!我恳求你们要像我一样,因为我曾一度也像你们一样。你们一点也没有亏负过我。
\textsuperscript{13}
你们知道:当我初次给你们宣讲福音时,正当我身患重病,
\textsuperscript{14}
虽然我的病势为你们是个试探,你们却没有轻看我,也没有厌弃我,反接待我有如一位天主的天使,有如\uline{基督}\uline{耶稣}。
\textsuperscript{15}
那么,你们当日所庆幸的在哪里呢?我敢为你们作证:如若可能,你们那时也会把你们的眼睛挖出来给我。
\footnote{由13节确知,\uline{保禄}在\uline{迦}省传福音时曾患重病,是否与\uwave{格}后12:7所说相同,无法断定。}
\textsuperscript{16}
那么,只因我给你们说实话,就成了你们的仇人吗?
\textsuperscript{17}
那些人对你们表示关心,并不怀好意;他们只是愿意使你们与我隔绝,好叫你们也关心他们。
\textsuperscript{18}
受人关心固然是好的,但应怀好意,且该常常如此,并不单是我在你们中间的时候。
\textsuperscript{19}
我的孩子们!我愿为你们再受产痛,直到\uline{基督}在你们内形成为止。
\textsuperscript{20}
恨不得我现今就在你们跟前,改变我的声调,因为我对你们实在放心不下。
\footnote{\uline{保禄}以母亲自比:为使\uline{迦拉达}人成为完善的基督徒,愿再受产痛之苦来苦心栽培他们。}

\textbf{基督徒才是自由的子女\quad}
\textsuperscript{21}
你们愿意属于法律的,请告诉我:你们没有听见法律说什么吗?
\textsuperscript{22}
法律曾记载说:\uline{亚巴郎}有两个儿子:一个生于婢女,一个生于自由的妇人。
\textsuperscript{23}
那生于婢女的,是按常例而生的;但那生于自由妇人的,却是因恩许而生的。
\textsuperscript{24}
这都含有寓意:那两个妇人是代表两个盟约:一是出于\uline{西乃}山,生子为奴,那即是\uline{哈加尔}——
\textsuperscript{25}
\uline{西乃}山是在\uline{阿剌伯}——这\uline{哈加尔}相当于现在的\uline{耶路撒冷},因为\uline{耶路撒冷}与她的子女同为奴隶。
\textsuperscript{26}
然而那属于天上的\uline{耶路撒冷}却是自由的,她就是我们的母亲:
\textsuperscript{27}
诚如经上记载说:“不生育的石女,喜乐吧!未经产痛的女人,欢呼高唱吧!因为被弃者的子女比有夫者的子女还多。”
\footnote{“婢女”\uline{哈加尔}是旧约或\uline{西乃}山盟约的预像;“自由的妇人”\uline{撒辣}是新约或\uline{熙雍}山盟约的预像。由婢女所生的儿子当然是奴隶,不能承继\uline{亚巴郎}的家业;同样,属旧约或\uline{西乃}山下盟约的人也是奴隶,对\uline{亚巴郎}那神妙的家业无权承继;只有由自由妇人所生的儿子,是恩许的儿子,有权承受天主对\uline{亚巴郎}所许的恩许。27节参阅\uwave{依}54:1。}
\textsuperscript{28}
弟兄们!你们像\uline{依撒格}一样,是恩许的子女。
\textsuperscript{29}
但是,先前那按常例而生的怎样迫害了那按神恩而生的,如今还是这样。
\textsuperscript{30}
然而经上说了什么?“你将婢女和她的儿子赶走,因为婢女的儿子不能与自由妇人的儿子,一同承受家业。”
\textsuperscript{31}
所以,弟兄们,我们不是婢女的子女,而是自由妇人的子女。
\footnote{\uline{保禄}说:信友如\uline{依撒格}一样是恩许的子女(28,31两节),如果是恩许的子女,那么,当然是自由的了,不应再受\uline{梅瑟}法律的束缚。29,30两节见\uwave{创}21:9-12。}

\begin{center}
	\textbf{\large{\songti 基督徒的正当生活}}
\end{center}

\textbf{第五章\quad恩许的自由决不可放弃\quad}
\textsuperscript{1}
\uline{基督}解救了我们,是为使我们获得自由;所以你们要站稳,不可再让奴隶的轭束缚住你们。
\textsuperscript{2}
请注意,我\uline{保禄}告诉你们:若你们还愿意受割损,\uline{基督}对你们就没有什么益处。
\textsuperscript{3}
我再向任何自愿受割损的人声明:他有遵守全部法律的义务。
\textsuperscript{4}
你们这些靠法律寻求成义的人,是与\uline{基督}断绝了关系,由恩宠上跌了下来。
\footnote{若有信友仍以割损为得救的条件,他就丧失了从\uline{基督}得来的自由及恩宠,而又成了奴隶及罪犯。}
\textsuperscript{5}
至于我们,我们却是依赖圣神,由于信德,怀着能成义的希望,
\textsuperscript{6}
因为在\uline{基督}\uline{耶稣}内,割损或不割损都算不得什么,唯有以爱德行事的信德,才算什么。
\footnote{如果我们成义是因信赖\uline{耶稣},那么割损与不割损,便自然没有什么价值了(\uwave{格}前7:19)。关于“以爱德行事的信德”,参阅\uwave{格}前13:2;\uwave{雅}2:18,22。}
\textsuperscript{7}
以前你们跑得好!有谁拦阻了你们去追随真理呢?
\textsuperscript{8}
这种劝诱,决不是出自那召选你们的天主。
\textsuperscript{9}
少许的酵母就能使整个面团发酵。
\textsuperscript{10}
我在主内信任你们,认为你们不会有什么别的心思;但那扰乱你们的人,不论他是谁,必要承受惩罚。
\textsuperscript{11}
至于我,弟兄们,如果我仍宣讲割损的需要,那我为什么还受迫害?若是这样,十字架的绊脚石就早已除去了。
\textsuperscript{12}
巴不得那些扰乱你们的人,将自己割净了!
\footnote{11节参阅\uwave{格}前1:23。12节“割净自己”,即“阉割”的意思。}

\textbf{自由是爱的表现\quad}
\textsuperscript{13}
弟兄们,你们蒙召选,是为得到自由;但不要以这自由作为放纵肉欲的藉口,惟要以爱德彼此服事。
\textsuperscript{14}
因为全部法律总括在这句话内:“爱你的近人如你自己。”
\textsuperscript{15}
但如果你们彼此相咬相吞,你们要小心,免得同归于尽。
\textsuperscript{16}
我告诉你们:你们若随圣神的引导行事,就决不会去满足本性的私欲,
\textsuperscript{17}
因为本性的私欲相反圣神的引导,圣神的引导相反本性的私欲:二者互相敌对,致使你们不能行你们所愿意的事。
\textsuperscript{18}
但如果你们随圣神的引导,就不在法律权下。
\textsuperscript{19}
本性私欲的作为是显而易见的:即淫乱、不洁、放荡、
\textsuperscript{20}
崇拜偶像、施行邪法、仇恨、竞争、嫉妒、忿怒、争吵、不睦、分党、
\textsuperscript{21}
妒恨、【凶杀】、醉酒、宴乐,以及与这些相类似的事。我以前劝戒过你们,如今再说一次:做这种事的人,决不能承受天主的国。
\textsuperscript{22}
然而圣神的效果却是:仁爱、喜乐、平安、忍耐、良善、温和、忠信、
\textsuperscript{23}
柔和、节制:关于这样的事,并没有法律禁止。
\textsuperscript{24}
凡属于\uline{耶稣}\uline{基督}的人,已把肉身同邪情和私欲钉在十字架上了。
\textsuperscript{25}
如果我们因圣神生活,就应随从圣神的引导而行事。
\textsuperscript{26}
不要贪图虚荣,不要彼此挑拨,互相嫉妒。
\footnote{真正的自由不是顺从肉欲去放肆,而是随从圣神去行事;不是以自私,而是以爱德为生活的原则。凡已属\uline{基督}的人,早已把肉身及一切私欲偏情钉在十字架上了。为这样的人,已不需要什么法律。14节参阅\uwave{肋}19:18;\uwave{罗}13:8-10。按\uline{拉丁}通行本,22,23两节中还有“宽宏、仁慈、贞洁”三种美德。}

\textbf{第六章\quad应彼此担待\quad}
\textsuperscript{1}
弟兄们,如果见一个人陷于某种过犯,你们既是属神的人,就该以柔和的心神矫正他;但你们自己要小心,免得也陷于诱惑。
\textsuperscript{2}
你们应彼此协助背负重担,这样,你们就满全了\uline{基督}的法律。
\textsuperscript{3}
人本来不算什么,若自以为算什么,就是欺骗自己。
\textsuperscript{4}
各人只该考验自己的行为,这样,对自己也许有可夸耀之处,但不是对别人夸耀,
\textsuperscript{5}
因为各人要背负自己的重担。
\textsuperscript{6}
学习真道的,应让教师分享自己的一切财物。
\footnote{\uline{保禄}劝勉信友要各人留心考验自己的缺点和毛病,如此,就必担待别人,不至于骄矜自夸。2节“\uline{基督}的法律”,即是指爱德的命令(\uwave{若}13:34)。}

\textbf{有其因必有其果\quad}
\textsuperscript{7}
你们切不要错了,天主是嘲笑不得的:人种什么,就收什么。
\textsuperscript{8}
那随从肉情撒种的,必由肉情收获败坏;然而那随从圣神撒种的,必由圣神收获永生。
\textsuperscript{9}
为此,我们行善不要厌倦;如果不松懈,到了适当的时节,必可收获。
\textsuperscript{10}
所以,我们一有机会,就应向众人行善,尤其应向有同样信德的家人。
\footnote{此处把人的一生比作撒种与收获,把肉身和心神(即本性和超性的精神)比作田地,人在这田地内播种工作;他若仅随肉身的好恶去行,必趋于败坏(丧亡);但若他随圣神的指引行事,必获得永生。10节,众信徒因有同样的信仰,故同是“天主的家人”(\uwave{弗}2:19)。}

\begin{center}
	\textbf{\large{\songti 结\quad论}}
\end{center}

\textbf{十字架是保禄的夸耀\quad}
\textsuperscript{11}
你们看,我亲手给你们写的是多么大的字!
\textsuperscript{12}
那些逼迫你们受割损的人,是想以外表的礼节来图人称赞,免得因\uline{基督}的十字架遭受迫害;
\textsuperscript{13}
其实,他们虽然受了割损,却也不遵守法律;他们只是愿意你们受割损,为能因在你们的肉身上所行的礼仪而夸耀。
\textsuperscript{14}
至于我,我只以我们的主\uline{耶稣}\uline{基督}的十字架来夸耀,因为藉着\uline{基督},世界于我已被钉在十字架上了;我于世界也被钉在十字架上了。
\textsuperscript{15}
其实,割损或不割损都算不得什么,要紧的是新受造的人。
\textsuperscript{16}
凡以此为规律而行的,愿平安与怜悯降在他们身上,即降在天主的新\uline{以色列}身上!
\textsuperscript{17}
从今以后,我切愿没有人再烦扰我,因为在我身上,我带有\uline{耶稣}的烙印。

\textbf{祝福辞\quad}
\textsuperscript{18}
弟兄们!愿我们的主\uline{耶稣}\uline{基督}的恩宠,常与你们的心灵同在!阿门。
\footnote{由11节可知:以上是\uline{保禄}所口授的,以下才是\uline{保禄}亲笔所写的,以证明本书出于自己,确实无伪。他末后用特大的字来写这段的用意,是把本书的主旨再提示出来,要人注意:割损与不割损都算不了什么,要紧的是在基督内成“一个新受造物”(\uwave{格}后5:17),因为唯有新的受造物,才是天主的新\uline{以色列},才是蒙受天主恩许的子民(\uwave{罗}2:29,9:24-26)。17节“\uline{耶稣}的烙印”是指\uline{保禄}为\uline{耶稣}受刑后所留下的伤痕;他以此当作他为宗徒的印号和证明(\uwave{格}后11:23-25)。}