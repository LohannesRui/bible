% 标题
\chapter*{若望二书}

% 右页眉
\rhead{若望二书(若二)}

% 正文
\textbf{致候辞\quad}
\textsuperscript{1}
我长老致书给蒙选的主母和她的子女,就是我在真理内所爱的,不但我一个人,而且也是所有认识真理的人所爱的;
\textsuperscript{2}
这爱的因由,就是那存在我们内,并永远与我们同在的真理。
\textsuperscript{3}
愿恩宠、仁爱与平安由天主父及天父之子\uline{耶稣}\uline{基督},在真理与爱情内与我们同在。
\renewcommand\thefootnote{\ding{\numexpr171+\value{footnote}}}
\footnote{此书与\uwave{若}三的作者自称为“长老”,决不是一位普通长老,而是\uline{亚细亚}众教会人人尽知,那号称为“长老”的\uline{若望}宗徒。他是当时仅存的一位宗徒,不但年纪老,而且德高望重。称收信者为“蒙选的主母和她的子女”,是指一教会及其信众(13节)。此短函的大旨和布局与\uwave{若}一相同。}

\textbf{应彼此相亲相爱\quad}
\textsuperscript{4}
我很喜欢,因为我遇见了你的一些子女,照我们由天父所领受的命令,在真理内生活。
\footnote{“在真理内生活”,即谓按福音的道理生活度日(\uwave{若}三3,4两节)。5节见\uwave{若}一2:7。}
\textsuperscript{5}
主母,我现在请求你,我们应该彼此相爱;这不是我写给你的一条新命令,而是我们从起初就有的命令。
\textsuperscript{6}
我们按照他的命令生活,这就是爱;你们应在爱中生活,这就是那命令,正如你们从起初听过的。

\textbf{应谨慎预防假学士\quad}
\textsuperscript{7}
的确,有许多迷惑人的,来到了世界上,他们不承认\uline{耶稣}\uline{基督}是在肉身内降世的;这样的人就是迷惑人的,就是假\uline{基督}。
\textsuperscript{8}
你们要谨慎,不要丧失你们劳苦所得的,反要领受圆满的赏报。
\textsuperscript{9}
凡是越规而不存在\uline{基督}道理内的,就没有天主;那存在这道理内的,这人有父也有子。
\footnote{“越规”是说离弃所传授的道理,而自创新说。“没有天主”是说丧失了与天主相通之恩。}
\textsuperscript{10}
若有人来到你们中,不带着这个道理,你们不要接他到家中,也不要向他请安,
\textsuperscript{11}
因为谁若向他请安,就是有分于他的邪恶工作。
\footnote{见\uwave{得}后3:6;\uline{铎}3:10。}

\textbf{结尾语\quad}
\textsuperscript{12}
虽然我还有许多事,要写给你们,但我不愿意用纸用墨,只希望到你们那里去,亲口面谈,好使我们的喜乐圆满无缺。
\textsuperscript{13}
你那蒙选的姊妹的子女问候你。
